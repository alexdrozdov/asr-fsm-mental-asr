\documentclass[fleqn,10pt,a4paper]{article}
\usepackage{ucs}
\usepackage[utf8x]{inputenc}
\usepackage[russian]{babel}
\usepackage{amsmath}
\usepackage{tikz}
\usepackage{wrapfig}

\usepackage{geometry}
\geometry{left=2cm}
\geometry{right=1.5cm}
\geometry{top=2cm}
\geometry{bottom=2cm}

\begin{document}
В этом разделе приведены расчеты, необходимые для определения направления
на одиночный источник звука. Направление на источник звука определяется на
основании разности хода фазового фронта до микрофонов решетки. В простейшем
случае решетка может содержать два микрофона.

Чертеж приведен на рисунке
рис.~\ref{single_s_src_dir}. Одиночный источник звука расположен в точке $D$ на расстоянии $L$ от центра плоскости микрофонной решетки. Микрофоны заркплены в точках $A$ и $B$, расстояние между микрофонами - $S$.

\begin{figure}[p]
\begin{tikzpicture}
\draw (0,0) --(4,6) -- (5,0) -- (0,0);
\draw (2.5,0) --(4,6) -- (4,0);
\node [left] at (0,0) {A};
\node [right] at (5,0) {B};
\node [below] at (4,0) {C};
\node [below] at (2.5,0) {K};
\node [above] at (4,6) {D};
\node [above left] at (3,2) {L};
\draw (2.55,0.3) to [out=0,in=90] (2.8,0);
\node [above right] at (2.7,0.1) {$\alpha$};
\draw [fill=black] (4,0) rectangle (4.15,0.15);
\node [above left]  at (2,3) {$l_1$};
\node [above right] at (4.3,4) {$l_2$};
\draw (0,0) circle (0.1);
\draw (5,0) circle (0.1);
\end{tikzpicture}\label{single_s_src_dir}
\caption{Прием сигнала от одиночного источника}
\end{figure}

\begin{equation*}
    \Delta{}l = l_1-l_2
\end{equation*}
\begin{equation*}
    l_1 = \sqrt{AC^2+CD^2},~~~l_2 = \sqrt{BC^2+CD^2}
\end{equation*}
\begin{equation*}
    CD = L\cdot\sin\alpha
\end{equation*}
\begin{equation*}
    AC=\frac{S}{2}+KC=\frac{S}{2}+L\cos\alpha
\end{equation*}
\begin{equation*}
    BC=\frac{S}{2}-KC=\frac{S}{2}-L\cos\alpha
\end{equation*}

\begin{equation*}
\begin{split}
	&\Delta{}l = l_1-l_2=\sqrt{AC^2+CD^2}-\sqrt{BC^2+CD^2}=\\
	&\sqrt{\left(\frac{S}{2}+L\cos\alpha\right)^2+L^2\cdot\sin^2\alpha}-
	\sqrt{\left(\frac{S}{2}-L\cos\alpha\right)^2+L^2\cdot\sin^2\alpha}=\\
	&\sqrt{\left(\frac{S^2}{4}+S\cdot{}L\cdot\cos\alpha+L^2\cdot\cos^2\alpha+L^2\cdot\sin^2\alpha\right)}-
	\sqrt{\left(\frac{S^2}{4}-S\cdot{}L\cdot\cos\alpha+L^2\cdot\cos^2\alpha+L^2\cdot\sin^2\alpha\right)}
\end{split}
\end{equation*}
\begin{equation}
	\boxed{
		\Delta{}l =
		\sqrt{\left(\frac{S^2}{4}+S\cdot{}L\cdot\cos\alpha+L^2\right)}-\sqrt{\left(\frac{S^2}{4}-S\cdot{}L\cdot\cos\alpha+L^2\right)}
	}\label{delta_l_cos_alpha}
\end{equation}

Выполняем замену
\begin{equation*}
	a = \frac{S^2}{4}+L^2,~~~x = S\cdot{}L\cdot\cos\alpha
\end{equation*}
\begin{equation*}
	\Delta{}l =	\sqrt{\left(a+x\right)}-\sqrt{\left(a-x\right)}
\end{equation*}

Решаем относительно $x$
\begin{equation*}
	{\Delta{}l}^2=a+x-2\sqrt{\left(a+x\right)}\cdot\sqrt{\left(a-x\right)}+a-x=2a-2\sqrt{\left(a+x\right)}\cdot\sqrt{\left(a-x\right)}=
\end{equation*}
\begin{equation*}
	2a-2\sqrt{a^2-x^2}
\end{equation*}
\begin{equation*}
	a^2-x^2=\left(a-\frac{{\Delta{}l}^2}{2}\right)^2
\end{equation*}
\begin{equation*}
	x = \sqrt{a^2-\left(a-\frac{{\Delta{}l}^2}{2}\right)^2}
\end{equation*}

Подставляем $a$ и $x$
\begin{equation*}
	S\cdot{}L\cdot\cos\alpha =
	\sqrt{\left(\frac{S^2}{4}+L^2\right)^2-\left(\frac{S^2}{4}+L^2-\frac{{\Delta{}l}^2}{2}\right)^2}
\end{equation*}
\begin{equation}
	\boxed{
		\cos\alpha =
		\frac{\sqrt{\left(\frac{S^2}{4}+L^2\right)^2-\left(\frac{S^2}{4}+L^2-\frac{{\Delta{}l}^2}{2}\right)^2}}{S\cdot{}L}
	}
	\label{cos_alpha_SL}
\end{equation}

Формула~(\ref{cos_alpha_SL}) позволяет рассчитать направление на источник звука
на основании разности хода фазового фронта $\Delta{}l$ и расстояния до источника
$L$. В реальных условиях расстояние $L$ неизвестно, поэтому точное определение
направления также невозможно. При расчетах по формуле~(\ref{cos_alpha_SL})
рекомендуется принять допущение, что $L\ge50\cdot{}S$. При этом погрешность в
определении направления зависит от величины $\Delta{}L=L-L_0$, где $L$ -
расстояние до источника, используемое при рассчетах, $L_0$ - истинное расстояние
до источника. Очевидно, что погрешность возрастает при увеличении значения
$\Delta{}L$.

На рисунке~\ref{single_s_src_err} приведен пояняющий чертеж. На чертеже
$\alpha$ --- направление, определенное по формуле~(\ref{cos_alpha_SL}) на
мнимый источник $D'$, $\alpha_0$ --- истинное направление на источник звука $D$.

\begin{figure}[p]
\begin{tikzpicture}
%\draw[help lines] (0,0) grid (5,6);
\draw (0,0) --(4,6) -- (5,0) -- (0,0);
\draw (2.5,0) --(4,6);
\draw (0,0) --(4,2) -- (5,0);
\draw (2.5,0) --(4,2);
\node [left] at (0,0) {A};
\node [right] at (5,0) {B};
\node [below] at (2.5,0) {K};
\node [above] at (4,6) {$D'$};
\node [above] at (4,2) {D};
\node [above left] at (3,2) {L};
\node [above right] at (3.3,0.8) {$L_0$};
\draw (2.55,0.3) to [out=0,in=90] (2.8,0);
\node [above right] at (2.8,0) {$\alpha$};
\draw (3.1,0.8) to [out=0,in=90] (3.7,0);
\node [above right] at (3.6,0.1) {$\alpha_0$};
\draw (0,0) circle (0.1);
\draw (5,0) circle (0.1);
\end{tikzpicture}\label{single_s_src_err}
\caption{Погрешность в определении направления при неизвестном $L_0$}
\end{figure}

Введем неравенство, связывающее между собой углы $\alpha$ и $\alpha_0$
максимальной погрешностью определения направления $\Delta\alpha$
\begin{equation*}
	|\alpha-\alpha_0| < \Delta\alpha
\end{equation*}
\begin{equation*}
	\left\{
		\begin{aligned}
			\alpha &< \alpha_0 + \Delta\alpha~,~\alpha>\alpha_0\\
			\alpha &> \alpha_0 - \Delta\alpha~,~\alpha<\alpha_0\\
		\end{aligned}
	\right.
\end{equation*}

Чертежу соответствует первое неравенство системы.
\begin{equation*}
			\alpha < \alpha_0 + \Delta\alpha
\end{equation*}
\begin{equation*}
			\alpha_0 > \alpha - \Delta\alpha
\end{equation*}

Необходимо наложить такое ограничение на $L_0$, чтобы выполнялись условия этого
неравенства.
\begin{equation*}
	\cos\alpha_0 < \cos\left(\alpha - \Delta\alpha\right)
\end{equation*}

Изменение знака неравенства обусловленно свойствами $\cos\alpha$.
Значение $\alpha$ получается на основе $\cos\alpha$ и должно рассчитываться по
формуле~(\ref{cos_alpha_SL}). Для удобства будет выполнена замена 
$\Delta{}l=\Delta{}l_0$. Расчет $\Delta{}l_0$ должен производиться для 
реального направления $\alpha_0$ и реального расстояния $L_0$ по 
формуле~\ref{delta_l_cos_alpha}. Для расчета $\cos\alpha_0$ также должна
применяться формула~(\ref{cos_alpha_SL}).

\begin{equation*}
	\frac{\sqrt{\left(\frac{S^2}{4}+{L_0}^2\right)^2-\left(\frac{S^2}{4}+{L_0}^2-\frac{{\Delta{}l_0}^2}{2}\right)^2}}{S\cdot{}L_0}
	< \cos\left(\alpha - \Delta\alpha\right)
\end{equation*}
\begin{equation*}
	\sqrt{\left(\frac{S^2}{4}+{L_0}^2\right)^2-\left(\frac{S^2}{4}+{L_0}^2-\frac{{\Delta{}l_0}^2}{2}\right)^2}
	< S\cdot{}L_0\cdot\cos\left(\alpha - \Delta\alpha\right)
\end{equation*}

Упрощаем подкоренное выражение

\begin{equation*}
\begin{split}
	&\left(\frac{S^2}{4}+{L_0}^2\right)^2-\left(\frac{S^2}{4}+{L_0}^2-\frac{{\Delta{}l_0}^2}{2}\right)^2=\\
	&\frac{{\Delta{}l_0}^2}{2}\cdot\left(2\frac{S^2}{4}+2{L_0}^2-\frac{{\Delta{}l_0}^2}{2}\right)=\\
	&\frac{{\Delta{}l_0}^2S^2}{4}+{\Delta{}l_0}^2{L_0}^2-\frac{{\Delta{}l_0}^4}{4}
\end{split}
\end{equation*}

Подставляем в неравенство

\begin{equation*}
	\sqrt{\frac{{\Delta{}l_0}^2S^2}{4}+{\Delta{}l_0}^2{L_0}^2-\frac{{\Delta{}l_0}^4}{4}}
	< S\cdot{}L_0\cdot\cos\left(\alpha - \Delta\alpha\right)
\end{equation*}
\begin{equation*}
	\frac{{\Delta{}l_0}^2S^2}{4}+{\Delta{}l_0}^2{L_0}^2-\frac{{\Delta{}l_0}^4}{4}
	< S^2\cdot{}{L_0}^2\cdot\cos^2\left(\alpha - \Delta\alpha\right),
\end{equation*}

при этом должно выполняться условие
$\frac{{\Delta{}l_0}^2S^2}{4}+{\Delta{}l_0}^2{L_0}^2-\frac{{\Delta{}l_0}^4}{4}\ge
0$. Т.к $S^2\cdot{}{L_0}^2\cdot\cos^2\left(\alpha - \Delta\alpha\right)\ge0$,
дополнительную проверку на неотрицательное значение подкоренного выражения можно
не выполнять.

\begin{equation*}
	{\Delta{}l_0}^2{L_0}^2-S^2\cdot{}{L_0}^2\cdot\cos^2\left(\alpha -
	\Delta\alpha\right) < \frac{{\Delta{}l_0}^4}{4} - \frac{{\Delta{}l_0}^2S^2}{4}
\end{equation*}
\begin{equation*}
	{L_0}^2\left({\Delta{}l_0}^2-S^2\cos^2\left(\alpha -
	\Delta\alpha\right)\right)
	< \frac{{\Delta{}l_0}^4 - {\Delta{}l_0}^2S^2}{4}
\end{equation*}
\begin{equation*}
	\left\{
		\begin{aligned}
			{L_0}^2 &< 
			\frac{{\Delta{}l_0}^4 - {\Delta{}l_0}^2S^2}{4\left({\Delta{}l_0}^2-
			S^2\cos^2\left(\alpha - \Delta\alpha\right)\right)},~
			{\Delta{}l_0}^2-S^2\cos^2\left(\alpha - \Delta\alpha\right)>0
			\\
			{L_0}^2 &>
			\frac{{\Delta{}l_0}^4 - {\Delta{}l_0}^2S^2}{4\left({\Delta{}l_0}^2-
			S^2\cos^2\left(\alpha - \Delta\alpha\right)\right)},~
			{\Delta{}l_0}^2-S^2\cos^2\left(\alpha - \Delta\alpha\right)<0
			\\
		\end{aligned}
	\right.
\end{equation*}

Проанализируем систему неравенств. Левые части неравенств представлены
положительным числом. Первое неравенство может выполняться только при условии,
что его правая сторона также положительное число. Ограничение на знаменатель
задано в условиях неравенства. Это означает, что числитель должен быть 
положительным. Проверим это условие.

\begin{equation*}
	{\Delta{}l_0}^4 - {\Delta{}l_0}^2S^2 =
	{\Delta{}l_0}^2\cdot\left({\Delta{}l_0}^2-S^2\right)
\end{equation*}

$S>\Delta{}l_0$ т.к. разность хода звуковых колебаний между двумя точками не
может превышать расстояния между этими точками. Отсюда

\begin{equation}
	\boxed{
		{\Delta{}l_0}^4-{\Delta{}l_0}^2S^2<0
	}\label{l_S_rel}
\end{equation}

Это означает, что первое неравенство системы невыполнимо.

Второе неравенство имеет отрицательный знаменатель в соответствии с условиями
этого неравенства и отрицательный числитель, как показано в (\ref{l_S_rel}). Это
означает, что правая часть этого неравенства строго неотрицательна, а значит,
существует множество значений $L_0$, для которых оно не выполняется и множество
значений $L_0$, для которых оно выполняется.

\begin{equation}
	\boxed {
		L_0 >
		\sqrt{\frac{{\Delta{}l_0}^4 - {\Delta{}l_0}^2S^2}{4\left({\Delta{}l_0}^2-
		S^2\cos^2\left(\alpha - \Delta\alpha\right)\right)}}
	}\label{L_delta_alpha}
\end{equation}

Определим условия, при которых знаменатель в подкоренном выражении
формулы~(\ref{L_delta_alpha}) принимает отрицательные значения.

\bigskip

\textbf{Алгоритм определения размера ближней зоны.}

1. Выбрать базовое значение $L_0$ из диапазона $5\ldots10 S$;

2. Выбрать максимальную погрешность $\Delta\alpha$;

3. Выбрать угол $\alpha_0$ для которого необходимо определить минимальное
расстояние;

4. Расчитать $\Delta_0$ по формуле~(\ref{delta_l_cos_alpha});

5. Рассчитать $\cos\alpha$ по формуле~(\ref{cos_alpha_SL}), подставив
$\Delta=\Delta_0$, определить~$\alpha$;

6. Рассчитать уточненное значение $L_0$ по формуле~(\ref{L_delta_alpha});

7. Пункты 3\ldots6 необходимо выполнить для всех интересующих углов.

\bigskip
Результат расчета размера ближней зоны, за пределами которой погрешность
опредления направления не превышает $2^\circ$ приведен на рисунке~\ref{nz_plot}.

\begin {figure}
	\begin{center}
		% GNUPLOT: LaTeX picture
\setlength{\unitlength}{0.240900pt}
\ifx\plotpoint\undefined\newsavebox{\plotpoint}\fi
\sbox{\plotpoint}{\rule[-0.200pt]{0.400pt}{0.400pt}}%
\begin{picture}(1500,900)(0,0)
\sbox{\plotpoint}{\rule[-0.200pt]{0.400pt}{0.400pt}}%
\put(170.0,173.0){\rule[-0.200pt]{308.111pt}{0.400pt}}
\put(170.0,173.0){\rule[-0.200pt]{4.818pt}{0.400pt}}
\put(150,173){\makebox(0,0)[r]{ 0}}
\put(1429.0,173.0){\rule[-0.200pt]{4.818pt}{0.400pt}}
\put(170.0,258.0){\rule[-0.200pt]{308.111pt}{0.400pt}}
\put(170.0,258.0){\rule[-0.200pt]{4.818pt}{0.400pt}}
\put(150,258){\makebox(0,0)[r]{ 0.02}}
\put(1429.0,258.0){\rule[-0.200pt]{4.818pt}{0.400pt}}
\put(170.0,344.0){\rule[-0.200pt]{308.111pt}{0.400pt}}
\put(170.0,344.0){\rule[-0.200pt]{4.818pt}{0.400pt}}
\put(150,344){\makebox(0,0)[r]{ 0.04}}
\put(1429.0,344.0){\rule[-0.200pt]{4.818pt}{0.400pt}}
\put(170.0,429.0){\rule[-0.200pt]{308.111pt}{0.400pt}}
\put(170.0,429.0){\rule[-0.200pt]{4.818pt}{0.400pt}}
\put(150,429){\makebox(0,0)[r]{ 0.06}}
\put(1429.0,429.0){\rule[-0.200pt]{4.818pt}{0.400pt}}
\put(170.0,514.0){\rule[-0.200pt]{308.111pt}{0.400pt}}
\put(170.0,514.0){\rule[-0.200pt]{4.818pt}{0.400pt}}
\put(150,514){\makebox(0,0)[r]{ 0.08}}
\put(1429.0,514.0){\rule[-0.200pt]{4.818pt}{0.400pt}}
\put(170.0,600.0){\rule[-0.200pt]{308.111pt}{0.400pt}}
\put(170.0,600.0){\rule[-0.200pt]{4.818pt}{0.400pt}}
\put(150,600){\makebox(0,0)[r]{ 0.1}}
\put(1429.0,600.0){\rule[-0.200pt]{4.818pt}{0.400pt}}
\put(170.0,685.0){\rule[-0.200pt]{308.111pt}{0.400pt}}
\put(170.0,685.0){\rule[-0.200pt]{4.818pt}{0.400pt}}
\put(150,685){\makebox(0,0)[r]{ 0.12}}
\put(1429.0,685.0){\rule[-0.200pt]{4.818pt}{0.400pt}}
\put(170.0,173.0){\rule[-0.200pt]{0.400pt}{123.341pt}}
\put(170.0,173.0){\rule[-0.200pt]{0.400pt}{4.818pt}}
\put(170,132){\makebox(0,0){ 0.15}}
\put(170.0,665.0){\rule[-0.200pt]{0.400pt}{4.818pt}}
\put(383.0,173.0){\rule[-0.200pt]{0.400pt}{123.341pt}}
\put(383.0,173.0){\rule[-0.200pt]{0.400pt}{4.818pt}}
\put(383,132){\makebox(0,0){ 0.1}}
\put(383.0,665.0){\rule[-0.200pt]{0.400pt}{4.818pt}}
\put(596.0,173.0){\rule[-0.200pt]{0.400pt}{123.341pt}}
\put(596.0,173.0){\rule[-0.200pt]{0.400pt}{4.818pt}}
\put(596,132){\makebox(0,0){ 0.05}}
\put(596.0,665.0){\rule[-0.200pt]{0.400pt}{4.818pt}}
\put(809.0,173.0){\rule[-0.200pt]{0.400pt}{123.341pt}}
\put(809.0,173.0){\rule[-0.200pt]{0.400pt}{4.818pt}}
\put(809,132){\makebox(0,0){ 0}}
\put(809.0,665.0){\rule[-0.200pt]{0.400pt}{4.818pt}}
\put(1023.0,173.0){\rule[-0.200pt]{0.400pt}{123.341pt}}
\put(1023.0,173.0){\rule[-0.200pt]{0.400pt}{4.818pt}}
\put(1023,132){\makebox(0,0){ 0.05}}
\put(1023.0,665.0){\rule[-0.200pt]{0.400pt}{4.818pt}}
\put(1236.0,173.0){\rule[-0.200pt]{0.400pt}{123.341pt}}
\put(1236.0,173.0){\rule[-0.200pt]{0.400pt}{4.818pt}}
\put(1236,132){\makebox(0,0){ 0.1}}
\put(1236.0,665.0){\rule[-0.200pt]{0.400pt}{4.818pt}}
\put(1449.0,173.0){\rule[-0.200pt]{0.400pt}{123.341pt}}
\put(1449.0,173.0){\rule[-0.200pt]{0.400pt}{4.818pt}}
\put(1449,132){\makebox(0,0){ 0.15}}
\put(1449.0,665.0){\rule[-0.200pt]{0.400pt}{4.818pt}}
\put(170.0,173.0){\rule[-0.200pt]{0.400pt}{123.341pt}}
\put(170.0,173.0){\rule[-0.200pt]{308.111pt}{0.400pt}}
\put(1449.0,173.0){\rule[-0.200pt]{0.400pt}{123.341pt}}
\put(170.0,685.0){\rule[-0.200pt]{308.111pt}{0.400pt}}
\put(809,747){\makebox(0,0){Границы ближней зоны}}
\put(1075,196){\usebox{\plotpoint}}
\multiput(1075.00,196.59)(1.332,0.485){11}{\rule{1.129pt}{0.117pt}}
\multiput(1075.00,195.17)(15.658,7.000){2}{\rule{0.564pt}{0.400pt}}
\multiput(1093.00,203.59)(1.332,0.485){11}{\rule{1.129pt}{0.117pt}}
\multiput(1093.00,202.17)(15.658,7.000){2}{\rule{0.564pt}{0.400pt}}
\multiput(1111.00,210.59)(1.022,0.488){13}{\rule{0.900pt}{0.117pt}}
\multiput(1111.00,209.17)(14.132,8.000){2}{\rule{0.450pt}{0.400pt}}
\multiput(1127.00,218.59)(0.956,0.488){13}{\rule{0.850pt}{0.117pt}}
\multiput(1127.00,217.17)(13.236,8.000){2}{\rule{0.425pt}{0.400pt}}
\multiput(1142.00,226.59)(0.890,0.488){13}{\rule{0.800pt}{0.117pt}}
\multiput(1142.00,225.17)(12.340,8.000){2}{\rule{0.400pt}{0.400pt}}
\multiput(1156.00,234.59)(0.728,0.489){15}{\rule{0.678pt}{0.118pt}}
\multiput(1156.00,233.17)(11.593,9.000){2}{\rule{0.339pt}{0.400pt}}
\multiput(1169.00,243.59)(0.669,0.489){15}{\rule{0.633pt}{0.118pt}}
\multiput(1169.00,242.17)(10.685,9.000){2}{\rule{0.317pt}{0.400pt}}
\multiput(1181.00,252.59)(0.611,0.489){15}{\rule{0.589pt}{0.118pt}}
\multiput(1181.00,251.17)(9.778,9.000){2}{\rule{0.294pt}{0.400pt}}
\multiput(1192.00,261.58)(0.547,0.491){17}{\rule{0.540pt}{0.118pt}}
\multiput(1192.00,260.17)(9.879,10.000){2}{\rule{0.270pt}{0.400pt}}
\multiput(1203.59,271.00)(0.489,0.553){15}{\rule{0.118pt}{0.544pt}}
\multiput(1202.17,271.00)(9.000,8.870){2}{\rule{0.400pt}{0.272pt}}
\multiput(1212.59,281.00)(0.489,0.553){15}{\rule{0.118pt}{0.544pt}}
\multiput(1211.17,281.00)(9.000,8.870){2}{\rule{0.400pt}{0.272pt}}
\multiput(1221.59,291.00)(0.488,0.626){13}{\rule{0.117pt}{0.600pt}}
\multiput(1220.17,291.00)(8.000,8.755){2}{\rule{0.400pt}{0.300pt}}
\multiput(1229.59,301.00)(0.488,0.692){13}{\rule{0.117pt}{0.650pt}}
\multiput(1228.17,301.00)(8.000,9.651){2}{\rule{0.400pt}{0.325pt}}
\multiput(1237.59,312.00)(0.482,0.943){9}{\rule{0.116pt}{0.833pt}}
\multiput(1236.17,312.00)(6.000,9.270){2}{\rule{0.400pt}{0.417pt}}
\multiput(1243.59,323.00)(0.482,0.852){9}{\rule{0.116pt}{0.767pt}}
\multiput(1242.17,323.00)(6.000,8.409){2}{\rule{0.400pt}{0.383pt}}
\multiput(1249.59,333.00)(0.482,0.943){9}{\rule{0.116pt}{0.833pt}}
\multiput(1248.17,333.00)(6.000,9.270){2}{\rule{0.400pt}{0.417pt}}
\multiput(1255.60,344.00)(0.468,1.505){5}{\rule{0.113pt}{1.200pt}}
\multiput(1254.17,344.00)(4.000,8.509){2}{\rule{0.400pt}{0.600pt}}
\multiput(1259.60,355.00)(0.468,1.505){5}{\rule{0.113pt}{1.200pt}}
\multiput(1258.17,355.00)(4.000,8.509){2}{\rule{0.400pt}{0.600pt}}
\multiput(1263.60,366.00)(0.468,1.505){5}{\rule{0.113pt}{1.200pt}}
\multiput(1262.17,366.00)(4.000,8.509){2}{\rule{0.400pt}{0.600pt}}
\multiput(1267.61,377.00)(0.447,2.248){3}{\rule{0.108pt}{1.567pt}}
\multiput(1266.17,377.00)(3.000,7.748){2}{\rule{0.400pt}{0.783pt}}
\put(1270.17,388){\rule{0.400pt}{2.300pt}}
\multiput(1269.17,388.00)(2.000,6.226){2}{\rule{0.400pt}{1.150pt}}
\put(1271.67,399){\rule{0.400pt}{2.650pt}}
\multiput(1271.17,399.00)(1.000,5.500){2}{\rule{0.400pt}{1.325pt}}
\put(1272.67,410){\rule{0.400pt}{2.409pt}}
\multiput(1272.17,410.00)(1.000,5.000){2}{\rule{0.400pt}{1.204pt}}
\put(1273.67,420){\rule{0.400pt}{2.650pt}}
\multiput(1273.17,420.00)(1.000,5.500){2}{\rule{0.400pt}{1.325pt}}
\put(1273.67,442){\rule{0.400pt}{2.409pt}}
\multiput(1274.17,442.00)(-1.000,5.000){2}{\rule{0.400pt}{1.204pt}}
\put(1272.67,452){\rule{0.400pt}{2.650pt}}
\multiput(1273.17,452.00)(-1.000,5.500){2}{\rule{0.400pt}{1.325pt}}
\put(1271.17,463){\rule{0.400pt}{2.100pt}}
\multiput(1272.17,463.00)(-2.000,5.641){2}{\rule{0.400pt}{1.050pt}}
\put(1269.17,473){\rule{0.400pt}{2.100pt}}
\multiput(1270.17,473.00)(-2.000,5.641){2}{\rule{0.400pt}{1.050pt}}
\multiput(1267.95,483.00)(-0.447,2.025){3}{\rule{0.108pt}{1.433pt}}
\multiput(1268.17,483.00)(-3.000,7.025){2}{\rule{0.400pt}{0.717pt}}
\multiput(1264.95,493.00)(-0.447,1.802){3}{\rule{0.108pt}{1.300pt}}
\multiput(1265.17,493.00)(-3.000,6.302){2}{\rule{0.400pt}{0.650pt}}
\multiput(1261.94,502.00)(-0.468,1.358){5}{\rule{0.113pt}{1.100pt}}
\multiput(1262.17,502.00)(-4.000,7.717){2}{\rule{0.400pt}{0.550pt}}
\multiput(1257.94,512.00)(-0.468,1.212){5}{\rule{0.113pt}{1.000pt}}
\multiput(1258.17,512.00)(-4.000,6.924){2}{\rule{0.400pt}{0.500pt}}
\multiput(1253.93,521.00)(-0.477,0.933){7}{\rule{0.115pt}{0.820pt}}
\multiput(1254.17,521.00)(-5.000,7.298){2}{\rule{0.400pt}{0.410pt}}
\multiput(1248.93,530.00)(-0.477,0.933){7}{\rule{0.115pt}{0.820pt}}
\multiput(1249.17,530.00)(-5.000,7.298){2}{\rule{0.400pt}{0.410pt}}
\multiput(1243.93,539.00)(-0.482,0.671){9}{\rule{0.116pt}{0.633pt}}
\multiput(1244.17,539.00)(-6.000,6.685){2}{\rule{0.400pt}{0.317pt}}
\multiput(1237.93,547.00)(-0.482,0.671){9}{\rule{0.116pt}{0.633pt}}
\multiput(1238.17,547.00)(-6.000,6.685){2}{\rule{0.400pt}{0.317pt}}
\multiput(1231.93,555.00)(-0.482,0.671){9}{\rule{0.116pt}{0.633pt}}
\multiput(1232.17,555.00)(-6.000,6.685){2}{\rule{0.400pt}{0.317pt}}
\multiput(1224.92,563.59)(-0.492,0.485){11}{\rule{0.500pt}{0.117pt}}
\multiput(1225.96,562.17)(-5.962,7.000){2}{\rule{0.250pt}{0.400pt}}
\multiput(1217.92,570.59)(-0.492,0.485){11}{\rule{0.500pt}{0.117pt}}
\multiput(1218.96,569.17)(-5.962,7.000){2}{\rule{0.250pt}{0.400pt}}
\multiput(1210.37,577.59)(-0.671,0.482){9}{\rule{0.633pt}{0.116pt}}
\multiput(1211.69,576.17)(-6.685,6.000){2}{\rule{0.317pt}{0.400pt}}
\multiput(1202.92,583.59)(-0.492,0.485){11}{\rule{0.500pt}{0.117pt}}
\multiput(1203.96,582.17)(-5.962,7.000){2}{\rule{0.250pt}{0.400pt}}
\multiput(1194.60,590.59)(-0.933,0.477){7}{\rule{0.820pt}{0.115pt}}
\multiput(1196.30,589.17)(-7.298,5.000){2}{\rule{0.410pt}{0.400pt}}
\multiput(1186.37,595.59)(-0.671,0.482){9}{\rule{0.633pt}{0.116pt}}
\multiput(1187.69,594.17)(-6.685,6.000){2}{\rule{0.317pt}{0.400pt}}
\multiput(1177.60,601.59)(-0.933,0.477){7}{\rule{0.820pt}{0.115pt}}
\multiput(1179.30,600.17)(-7.298,5.000){2}{\rule{0.410pt}{0.400pt}}
\multiput(1167.85,606.60)(-1.212,0.468){5}{\rule{1.000pt}{0.113pt}}
\multiput(1169.92,605.17)(-6.924,4.000){2}{\rule{0.500pt}{0.400pt}}
\multiput(1158.85,610.60)(-1.212,0.468){5}{\rule{1.000pt}{0.113pt}}
\multiput(1160.92,609.17)(-6.924,4.000){2}{\rule{0.500pt}{0.400pt}}
\multiput(1149.43,614.60)(-1.358,0.468){5}{\rule{1.100pt}{0.113pt}}
\multiput(1151.72,613.17)(-7.717,4.000){2}{\rule{0.550pt}{0.400pt}}
\multiput(1138.05,618.61)(-2.025,0.447){3}{\rule{1.433pt}{0.108pt}}
\multiput(1141.03,617.17)(-7.025,3.000){2}{\rule{0.717pt}{0.400pt}}
\put(1124,621.17){\rule{2.100pt}{0.400pt}}
\multiput(1129.64,620.17)(-5.641,2.000){2}{\rule{1.050pt}{0.400pt}}
\put(1114,623.17){\rule{2.100pt}{0.400pt}}
\multiput(1119.64,622.17)(-5.641,2.000){2}{\rule{1.050pt}{0.400pt}}
\put(1104,625.17){\rule{2.100pt}{0.400pt}}
\multiput(1109.64,624.17)(-5.641,2.000){2}{\rule{1.050pt}{0.400pt}}
\put(1093,626.67){\rule{2.650pt}{0.400pt}}
\multiput(1098.50,626.17)(-5.500,1.000){2}{\rule{1.325pt}{0.400pt}}
\put(1275.0,431.0){\rule[-0.200pt]{0.400pt}{2.650pt}}
\put(1061,626.67){\rule{2.650pt}{0.400pt}}
\multiput(1066.50,627.17)(-5.500,-1.000){2}{\rule{1.325pt}{0.400pt}}
\put(1050,625.67){\rule{2.650pt}{0.400pt}}
\multiput(1055.50,626.17)(-5.500,-1.000){2}{\rule{1.325pt}{0.400pt}}
\put(1039,624.17){\rule{2.300pt}{0.400pt}}
\multiput(1045.23,625.17)(-6.226,-2.000){2}{\rule{1.150pt}{0.400pt}}
\put(1028,622.17){\rule{2.300pt}{0.400pt}}
\multiput(1034.23,623.17)(-6.226,-2.000){2}{\rule{1.150pt}{0.400pt}}
\multiput(1021.50,620.95)(-2.248,-0.447){3}{\rule{1.567pt}{0.108pt}}
\multiput(1024.75,621.17)(-7.748,-3.000){2}{\rule{0.783pt}{0.400pt}}
\multiput(1012.02,617.94)(-1.505,-0.468){5}{\rule{1.200pt}{0.113pt}}
\multiput(1014.51,618.17)(-8.509,-4.000){2}{\rule{0.600pt}{0.400pt}}
\multiput(1001.02,613.94)(-1.505,-0.468){5}{\rule{1.200pt}{0.113pt}}
\multiput(1003.51,614.17)(-8.509,-4.000){2}{\rule{0.600pt}{0.400pt}}
\multiput(990.93,609.93)(-1.155,-0.477){7}{\rule{0.980pt}{0.115pt}}
\multiput(992.97,610.17)(-8.966,-5.000){2}{\rule{0.490pt}{0.400pt}}
\multiput(980.54,604.93)(-0.943,-0.482){9}{\rule{0.833pt}{0.116pt}}
\multiput(982.27,605.17)(-9.270,-6.000){2}{\rule{0.417pt}{0.400pt}}
\multiput(969.82,598.93)(-0.852,-0.482){9}{\rule{0.767pt}{0.116pt}}
\multiput(971.41,599.17)(-8.409,-6.000){2}{\rule{0.383pt}{0.400pt}}
\multiput(959.98,592.93)(-0.798,-0.485){11}{\rule{0.729pt}{0.117pt}}
\multiput(961.49,593.17)(-9.488,-7.000){2}{\rule{0.364pt}{0.400pt}}
\multiput(949.30,585.93)(-0.692,-0.488){13}{\rule{0.650pt}{0.117pt}}
\multiput(950.65,586.17)(-9.651,-8.000){2}{\rule{0.325pt}{0.400pt}}
\multiput(938.74,577.93)(-0.553,-0.489){15}{\rule{0.544pt}{0.118pt}}
\multiput(939.87,578.17)(-8.870,-9.000){2}{\rule{0.272pt}{0.400pt}}
\multiput(928.74,568.93)(-0.553,-0.489){15}{\rule{0.544pt}{0.118pt}}
\multiput(929.87,569.17)(-8.870,-9.000){2}{\rule{0.272pt}{0.400pt}}
\multiput(918.92,559.92)(-0.495,-0.491){17}{\rule{0.500pt}{0.118pt}}
\multiput(919.96,560.17)(-8.962,-10.000){2}{\rule{0.250pt}{0.400pt}}
\multiput(909.92,548.76)(-0.491,-0.547){17}{\rule{0.118pt}{0.540pt}}
\multiput(910.17,549.88)(-10.000,-9.879){2}{\rule{0.400pt}{0.270pt}}
\multiput(899.92,537.59)(-0.491,-0.600){17}{\rule{0.118pt}{0.580pt}}
\multiput(900.17,538.80)(-10.000,-10.796){2}{\rule{0.400pt}{0.290pt}}
\multiput(889.93,525.19)(-0.489,-0.728){15}{\rule{0.118pt}{0.678pt}}
\multiput(890.17,526.59)(-9.000,-11.593){2}{\rule{0.400pt}{0.339pt}}
\multiput(880.93,512.00)(-0.489,-0.786){15}{\rule{0.118pt}{0.722pt}}
\multiput(881.17,513.50)(-9.000,-12.501){2}{\rule{0.400pt}{0.361pt}}
\multiput(871.93,497.47)(-0.488,-0.956){13}{\rule{0.117pt}{0.850pt}}
\multiput(872.17,499.24)(-8.000,-13.236){2}{\rule{0.400pt}{0.425pt}}
\multiput(863.93,482.26)(-0.488,-1.022){13}{\rule{0.117pt}{0.900pt}}
\multiput(864.17,484.13)(-8.000,-14.132){2}{\rule{0.400pt}{0.450pt}}
\multiput(855.93,466.06)(-0.488,-1.088){13}{\rule{0.117pt}{0.950pt}}
\multiput(856.17,468.03)(-8.000,-15.028){2}{\rule{0.400pt}{0.475pt}}
\multiput(847.93,448.43)(-0.488,-1.286){13}{\rule{0.117pt}{1.100pt}}
\multiput(848.17,450.72)(-8.000,-17.717){2}{\rule{0.400pt}{0.550pt}}
\multiput(839.93,426.77)(-0.482,-1.847){9}{\rule{0.116pt}{1.500pt}}
\multiput(840.17,429.89)(-6.000,-17.887){2}{\rule{0.400pt}{0.750pt}}
\multiput(833.93,406.13)(-0.485,-1.713){11}{\rule{0.117pt}{1.414pt}}
\multiput(834.17,409.06)(-7.000,-20.065){2}{\rule{0.400pt}{0.707pt}}
\multiput(826.93,379.95)(-0.477,-2.825){7}{\rule{0.115pt}{2.180pt}}
\multiput(827.17,384.48)(-5.000,-21.475){2}{\rule{0.400pt}{1.090pt}}
\multiput(821.93,352.29)(-0.477,-3.382){7}{\rule{0.115pt}{2.580pt}}
\multiput(822.17,357.65)(-5.000,-25.645){2}{\rule{0.400pt}{1.290pt}}
\multiput(816.94,316.64)(-0.468,-5.160){5}{\rule{0.113pt}{3.700pt}}
\multiput(817.17,324.32)(-4.000,-28.320){2}{\rule{0.400pt}{1.850pt}}
\multiput(812.95,269.02)(-0.447,-10.509){3}{\rule{0.108pt}{6.500pt}}
\multiput(813.17,282.51)(-3.000,-34.509){2}{\rule{0.400pt}{3.250pt}}
\put(809.67,173){\rule{0.400pt}{18.067pt}}
\multiput(810.17,210.50)(-1.000,-37.500){2}{\rule{0.400pt}{9.034pt}}
\put(808.17,173){\rule{0.400pt}{15.100pt}}
\multiput(809.17,173.00)(-2.000,43.659){2}{\rule{0.400pt}{7.550pt}}
\multiput(806.95,248.00)(-0.447,10.509){3}{\rule{0.108pt}{6.500pt}}
\multiput(807.17,248.00)(-3.000,34.509){2}{\rule{0.400pt}{3.250pt}}
\multiput(803.94,296.00)(-0.468,5.160){5}{\rule{0.113pt}{3.700pt}}
\multiput(804.17,296.00)(-4.000,28.320){2}{\rule{0.400pt}{1.850pt}}
\multiput(799.93,332.00)(-0.477,3.382){7}{\rule{0.115pt}{2.580pt}}
\multiput(800.17,332.00)(-5.000,25.645){2}{\rule{0.400pt}{1.290pt}}
\multiput(794.93,363.00)(-0.477,2.825){7}{\rule{0.115pt}{2.180pt}}
\multiput(795.17,363.00)(-5.000,21.475){2}{\rule{0.400pt}{1.090pt}}
\multiput(789.93,389.00)(-0.485,1.713){11}{\rule{0.117pt}{1.414pt}}
\multiput(790.17,389.00)(-7.000,20.065){2}{\rule{0.400pt}{0.707pt}}
\multiput(782.93,412.00)(-0.482,1.847){9}{\rule{0.116pt}{1.500pt}}
\multiput(783.17,412.00)(-6.000,17.887){2}{\rule{0.400pt}{0.750pt}}
\multiput(776.93,433.00)(-0.488,1.286){13}{\rule{0.117pt}{1.100pt}}
\multiput(777.17,433.00)(-8.000,17.717){2}{\rule{0.400pt}{0.550pt}}
\multiput(768.93,453.00)(-0.488,1.088){13}{\rule{0.117pt}{0.950pt}}
\multiput(769.17,453.00)(-8.000,15.028){2}{\rule{0.400pt}{0.475pt}}
\multiput(760.93,470.00)(-0.488,1.022){13}{\rule{0.117pt}{0.900pt}}
\multiput(761.17,470.00)(-8.000,14.132){2}{\rule{0.400pt}{0.450pt}}
\multiput(752.93,486.00)(-0.488,0.956){13}{\rule{0.117pt}{0.850pt}}
\multiput(753.17,486.00)(-8.000,13.236){2}{\rule{0.400pt}{0.425pt}}
\multiput(744.93,501.00)(-0.489,0.786){15}{\rule{0.118pt}{0.722pt}}
\multiput(745.17,501.00)(-9.000,12.501){2}{\rule{0.400pt}{0.361pt}}
\multiput(735.93,515.00)(-0.489,0.728){15}{\rule{0.118pt}{0.678pt}}
\multiput(736.17,515.00)(-9.000,11.593){2}{\rule{0.400pt}{0.339pt}}
\multiput(726.92,528.00)(-0.491,0.600){17}{\rule{0.118pt}{0.580pt}}
\multiput(727.17,528.00)(-10.000,10.796){2}{\rule{0.400pt}{0.290pt}}
\multiput(716.92,540.00)(-0.491,0.547){17}{\rule{0.118pt}{0.540pt}}
\multiput(717.17,540.00)(-10.000,9.879){2}{\rule{0.400pt}{0.270pt}}
\multiput(705.92,551.58)(-0.495,0.491){17}{\rule{0.500pt}{0.118pt}}
\multiput(706.96,550.17)(-8.962,10.000){2}{\rule{0.250pt}{0.400pt}}
\multiput(695.74,561.59)(-0.553,0.489){15}{\rule{0.544pt}{0.118pt}}
\multiput(696.87,560.17)(-8.870,9.000){2}{\rule{0.272pt}{0.400pt}}
\multiput(685.74,570.59)(-0.553,0.489){15}{\rule{0.544pt}{0.118pt}}
\multiput(686.87,569.17)(-8.870,9.000){2}{\rule{0.272pt}{0.400pt}}
\multiput(675.30,579.59)(-0.692,0.488){13}{\rule{0.650pt}{0.117pt}}
\multiput(676.65,578.17)(-9.651,8.000){2}{\rule{0.325pt}{0.400pt}}
\multiput(663.98,587.59)(-0.798,0.485){11}{\rule{0.729pt}{0.117pt}}
\multiput(665.49,586.17)(-9.488,7.000){2}{\rule{0.364pt}{0.400pt}}
\multiput(652.82,594.59)(-0.852,0.482){9}{\rule{0.767pt}{0.116pt}}
\multiput(654.41,593.17)(-8.409,6.000){2}{\rule{0.383pt}{0.400pt}}
\multiput(642.54,600.59)(-0.943,0.482){9}{\rule{0.833pt}{0.116pt}}
\multiput(644.27,599.17)(-9.270,6.000){2}{\rule{0.417pt}{0.400pt}}
\multiput(630.93,606.59)(-1.155,0.477){7}{\rule{0.980pt}{0.115pt}}
\multiput(632.97,605.17)(-8.966,5.000){2}{\rule{0.490pt}{0.400pt}}
\multiput(619.02,611.60)(-1.505,0.468){5}{\rule{1.200pt}{0.113pt}}
\multiput(621.51,610.17)(-8.509,4.000){2}{\rule{0.600pt}{0.400pt}}
\multiput(608.02,615.60)(-1.505,0.468){5}{\rule{1.200pt}{0.113pt}}
\multiput(610.51,614.17)(-8.509,4.000){2}{\rule{0.600pt}{0.400pt}}
\multiput(595.50,619.61)(-2.248,0.447){3}{\rule{1.567pt}{0.108pt}}
\multiput(598.75,618.17)(-7.748,3.000){2}{\rule{0.783pt}{0.400pt}}
\put(580,622.17){\rule{2.300pt}{0.400pt}}
\multiput(586.23,621.17)(-6.226,2.000){2}{\rule{1.150pt}{0.400pt}}
\put(569,624.17){\rule{2.300pt}{0.400pt}}
\multiput(575.23,623.17)(-6.226,2.000){2}{\rule{1.150pt}{0.400pt}}
\put(558,625.67){\rule{2.650pt}{0.400pt}}
\multiput(563.50,625.17)(-5.500,1.000){2}{\rule{1.325pt}{0.400pt}}
\put(547,626.67){\rule{2.650pt}{0.400pt}}
\multiput(552.50,626.17)(-5.500,1.000){2}{\rule{1.325pt}{0.400pt}}
\put(1072.0,628.0){\rule[-0.200pt]{5.059pt}{0.400pt}}
\put(515,626.67){\rule{2.650pt}{0.400pt}}
\multiput(520.50,627.17)(-5.500,-1.000){2}{\rule{1.325pt}{0.400pt}}
\put(505,625.17){\rule{2.100pt}{0.400pt}}
\multiput(510.64,626.17)(-5.641,-2.000){2}{\rule{1.050pt}{0.400pt}}
\put(495,623.17){\rule{2.100pt}{0.400pt}}
\multiput(500.64,624.17)(-5.641,-2.000){2}{\rule{1.050pt}{0.400pt}}
\put(485,621.17){\rule{2.100pt}{0.400pt}}
\multiput(490.64,622.17)(-5.641,-2.000){2}{\rule{1.050pt}{0.400pt}}
\multiput(479.05,619.95)(-2.025,-0.447){3}{\rule{1.433pt}{0.108pt}}
\multiput(482.03,620.17)(-7.025,-3.000){2}{\rule{0.717pt}{0.400pt}}
\multiput(470.43,616.94)(-1.358,-0.468){5}{\rule{1.100pt}{0.113pt}}
\multiput(472.72,617.17)(-7.717,-4.000){2}{\rule{0.550pt}{0.400pt}}
\multiput(460.85,612.94)(-1.212,-0.468){5}{\rule{1.000pt}{0.113pt}}
\multiput(462.92,613.17)(-6.924,-4.000){2}{\rule{0.500pt}{0.400pt}}
\multiput(451.85,608.94)(-1.212,-0.468){5}{\rule{1.000pt}{0.113pt}}
\multiput(453.92,609.17)(-6.924,-4.000){2}{\rule{0.500pt}{0.400pt}}
\multiput(443.60,604.93)(-0.933,-0.477){7}{\rule{0.820pt}{0.115pt}}
\multiput(445.30,605.17)(-7.298,-5.000){2}{\rule{0.410pt}{0.400pt}}
\multiput(435.37,599.93)(-0.671,-0.482){9}{\rule{0.633pt}{0.116pt}}
\multiput(436.69,600.17)(-6.685,-6.000){2}{\rule{0.317pt}{0.400pt}}
\multiput(426.60,593.93)(-0.933,-0.477){7}{\rule{0.820pt}{0.115pt}}
\multiput(428.30,594.17)(-7.298,-5.000){2}{\rule{0.410pt}{0.400pt}}
\multiput(418.92,588.93)(-0.492,-0.485){11}{\rule{0.500pt}{0.117pt}}
\multiput(419.96,589.17)(-5.962,-7.000){2}{\rule{0.250pt}{0.400pt}}
\multiput(411.37,581.93)(-0.671,-0.482){9}{\rule{0.633pt}{0.116pt}}
\multiput(412.69,582.17)(-6.685,-6.000){2}{\rule{0.317pt}{0.400pt}}
\multiput(403.92,575.93)(-0.492,-0.485){11}{\rule{0.500pt}{0.117pt}}
\multiput(404.96,576.17)(-5.962,-7.000){2}{\rule{0.250pt}{0.400pt}}
\multiput(396.92,568.93)(-0.492,-0.485){11}{\rule{0.500pt}{0.117pt}}
\multiput(397.96,569.17)(-5.962,-7.000){2}{\rule{0.250pt}{0.400pt}}
\multiput(390.93,560.37)(-0.482,-0.671){9}{\rule{0.116pt}{0.633pt}}
\multiput(391.17,561.69)(-6.000,-6.685){2}{\rule{0.400pt}{0.317pt}}
\multiput(384.93,552.37)(-0.482,-0.671){9}{\rule{0.116pt}{0.633pt}}
\multiput(385.17,553.69)(-6.000,-6.685){2}{\rule{0.400pt}{0.317pt}}
\multiput(378.93,544.37)(-0.482,-0.671){9}{\rule{0.116pt}{0.633pt}}
\multiput(379.17,545.69)(-6.000,-6.685){2}{\rule{0.400pt}{0.317pt}}
\multiput(372.93,535.60)(-0.477,-0.933){7}{\rule{0.115pt}{0.820pt}}
\multiput(373.17,537.30)(-5.000,-7.298){2}{\rule{0.400pt}{0.410pt}}
\multiput(367.93,526.60)(-0.477,-0.933){7}{\rule{0.115pt}{0.820pt}}
\multiput(368.17,528.30)(-5.000,-7.298){2}{\rule{0.400pt}{0.410pt}}
\multiput(362.94,516.85)(-0.468,-1.212){5}{\rule{0.113pt}{1.000pt}}
\multiput(363.17,518.92)(-4.000,-6.924){2}{\rule{0.400pt}{0.500pt}}
\multiput(358.94,507.43)(-0.468,-1.358){5}{\rule{0.113pt}{1.100pt}}
\multiput(359.17,509.72)(-4.000,-7.717){2}{\rule{0.400pt}{0.550pt}}
\multiput(354.95,496.60)(-0.447,-1.802){3}{\rule{0.108pt}{1.300pt}}
\multiput(355.17,499.30)(-3.000,-6.302){2}{\rule{0.400pt}{0.650pt}}
\multiput(351.95,487.05)(-0.447,-2.025){3}{\rule{0.108pt}{1.433pt}}
\multiput(352.17,490.03)(-3.000,-7.025){2}{\rule{0.400pt}{0.717pt}}
\put(348.17,473){\rule{0.400pt}{2.100pt}}
\multiput(349.17,478.64)(-2.000,-5.641){2}{\rule{0.400pt}{1.050pt}}
\put(346.17,463){\rule{0.400pt}{2.100pt}}
\multiput(347.17,468.64)(-2.000,-5.641){2}{\rule{0.400pt}{1.050pt}}
\put(344.67,452){\rule{0.400pt}{2.650pt}}
\multiput(345.17,457.50)(-1.000,-5.500){2}{\rule{0.400pt}{1.325pt}}
\put(343.67,442){\rule{0.400pt}{2.409pt}}
\multiput(344.17,447.00)(-1.000,-5.000){2}{\rule{0.400pt}{1.204pt}}
\put(526.0,628.0){\rule[-0.200pt]{5.059pt}{0.400pt}}
\put(343.67,420){\rule{0.400pt}{2.650pt}}
\multiput(343.17,425.50)(1.000,-5.500){2}{\rule{0.400pt}{1.325pt}}
\put(344.67,410){\rule{0.400pt}{2.409pt}}
\multiput(344.17,415.00)(1.000,-5.000){2}{\rule{0.400pt}{1.204pt}}
\put(345.67,399){\rule{0.400pt}{2.650pt}}
\multiput(345.17,404.50)(1.000,-5.500){2}{\rule{0.400pt}{1.325pt}}
\put(347.17,388){\rule{0.400pt}{2.300pt}}
\multiput(346.17,394.23)(2.000,-6.226){2}{\rule{0.400pt}{1.150pt}}
\multiput(349.61,381.50)(0.447,-2.248){3}{\rule{0.108pt}{1.567pt}}
\multiput(348.17,384.75)(3.000,-7.748){2}{\rule{0.400pt}{0.783pt}}
\multiput(352.60,372.02)(0.468,-1.505){5}{\rule{0.113pt}{1.200pt}}
\multiput(351.17,374.51)(4.000,-8.509){2}{\rule{0.400pt}{0.600pt}}
\multiput(356.60,361.02)(0.468,-1.505){5}{\rule{0.113pt}{1.200pt}}
\multiput(355.17,363.51)(4.000,-8.509){2}{\rule{0.400pt}{0.600pt}}
\multiput(360.60,350.02)(0.468,-1.505){5}{\rule{0.113pt}{1.200pt}}
\multiput(359.17,352.51)(4.000,-8.509){2}{\rule{0.400pt}{0.600pt}}
\multiput(364.59,340.54)(0.482,-0.943){9}{\rule{0.116pt}{0.833pt}}
\multiput(363.17,342.27)(6.000,-9.270){2}{\rule{0.400pt}{0.417pt}}
\multiput(370.59,329.82)(0.482,-0.852){9}{\rule{0.116pt}{0.767pt}}
\multiput(369.17,331.41)(6.000,-8.409){2}{\rule{0.400pt}{0.383pt}}
\multiput(376.59,319.54)(0.482,-0.943){9}{\rule{0.116pt}{0.833pt}}
\multiput(375.17,321.27)(6.000,-9.270){2}{\rule{0.400pt}{0.417pt}}
\multiput(382.59,309.30)(0.488,-0.692){13}{\rule{0.117pt}{0.650pt}}
\multiput(381.17,310.65)(8.000,-9.651){2}{\rule{0.400pt}{0.325pt}}
\multiput(390.59,298.51)(0.488,-0.626){13}{\rule{0.117pt}{0.600pt}}
\multiput(389.17,299.75)(8.000,-8.755){2}{\rule{0.400pt}{0.300pt}}
\multiput(398.59,288.74)(0.489,-0.553){15}{\rule{0.118pt}{0.544pt}}
\multiput(397.17,289.87)(9.000,-8.870){2}{\rule{0.400pt}{0.272pt}}
\multiput(407.59,278.74)(0.489,-0.553){15}{\rule{0.118pt}{0.544pt}}
\multiput(406.17,279.87)(9.000,-8.870){2}{\rule{0.400pt}{0.272pt}}
\multiput(416.00,269.92)(0.547,-0.491){17}{\rule{0.540pt}{0.118pt}}
\multiput(416.00,270.17)(9.879,-10.000){2}{\rule{0.270pt}{0.400pt}}
\multiput(427.00,259.93)(0.611,-0.489){15}{\rule{0.589pt}{0.118pt}}
\multiput(427.00,260.17)(9.778,-9.000){2}{\rule{0.294pt}{0.400pt}}
\multiput(438.00,250.93)(0.669,-0.489){15}{\rule{0.633pt}{0.118pt}}
\multiput(438.00,251.17)(10.685,-9.000){2}{\rule{0.317pt}{0.400pt}}
\multiput(450.00,241.93)(0.728,-0.489){15}{\rule{0.678pt}{0.118pt}}
\multiput(450.00,242.17)(11.593,-9.000){2}{\rule{0.339pt}{0.400pt}}
\multiput(463.00,232.93)(0.890,-0.488){13}{\rule{0.800pt}{0.117pt}}
\multiput(463.00,233.17)(12.340,-8.000){2}{\rule{0.400pt}{0.400pt}}
\multiput(477.00,224.93)(0.956,-0.488){13}{\rule{0.850pt}{0.117pt}}
\multiput(477.00,225.17)(13.236,-8.000){2}{\rule{0.425pt}{0.400pt}}
\multiput(492.00,216.93)(1.022,-0.488){13}{\rule{0.900pt}{0.117pt}}
\multiput(492.00,217.17)(14.132,-8.000){2}{\rule{0.450pt}{0.400pt}}
\multiput(508.00,208.93)(1.332,-0.485){11}{\rule{1.129pt}{0.117pt}}
\multiput(508.00,209.17)(15.658,-7.000){2}{\rule{0.564pt}{0.400pt}}
\multiput(526.00,201.93)(1.332,-0.485){11}{\rule{1.129pt}{0.117pt}}
\multiput(526.00,202.17)(15.658,-7.000){2}{\rule{0.564pt}{0.400pt}}
\put(344.0,431.0){\rule[-0.200pt]{0.400pt}{2.650pt}}
\put(170.0,173.0){\rule[-0.200pt]{0.400pt}{123.341pt}}
\put(170.0,173.0){\rule[-0.200pt]{308.111pt}{0.400pt}}
\put(1449.0,173.0){\rule[-0.200pt]{0.400pt}{123.341pt}}
\put(170.0,685.0){\rule[-0.200pt]{308.111pt}{0.400pt}}
\end{picture}

	\end{center}\label{nz_plot}
	\caption{Граница ближней зоны при погрешности определения угла $2^\circ$}
\end {figure}

\bigskip
\textbf{Определение направления на одиночный источник звука}

Физическая основа определения направления на источник звука --- разность хода
фазового фронта от источника к нескольким микрофонам. Определив разность хода
можно рассчитать направление на источник. В этой работе рассматривается два
способа:

--- Корреляция между сигналами с микрофонов

--- Искажения спектра при суммировании сигналов.

Рассмотрим искажения спектра подробнее. Пусть в системе установлено два
микрофона на расстоянии $S$ друг от друга. Источник смещен относительно нормали
к плоскости микрофонов. Для начала будем считать, что источник формирует звук на
одной единственной частоте $\omega$.

Сигналы на каждом из микрофонов:
\begin{equation*}
	s_1(t) = \sin\left(\omega{}t+\phi_1\right)
\end{equation*}
\begin{equation*}
	s_2(t) = \sin\left(\omega{}t+\phi_2\right)
\end{equation*}

Просуммируем эти сигналы (как это происходит в антенной решетке)
\begin{equation*}
	\begin{aligned}
		s_1(t) + s_2(t) &=
		\sin\left(\omega{}t+\phi_1\right)+\sin\left(\omega{}t+\phi_2\right) \\
		&= 2\sin\frac{\omega{}t+\phi_1+\omega{}t+\phi_2}{2}\cdot
		\cos\frac{\omega{}t+\phi_1-\omega{}t-\phi_2}{2} = \\
		&= 2\sin\left(\omega{}t+\frac{\phi_1+\phi_2}{2}\right)
		\cos\frac{\phi_1-\phi_2}{2} = \\
		&= 2\sin\left(\omega{}t+\frac{\phi_1+\phi_2}{2}\right)
		\cos\frac{\Delta\phi}{2}
	\end{aligned}
\end{equation*}
\begin{equation*}
	\Delta\phi=\frac{\Delta{}l\cdot\omega}{C}=2\pi\Delta{}l\frac{f}{C}
\end{equation*}
\begin{equation*}
	s_1(t) + s_2(t) = 2\sin\left(\omega{}t+\frac{\phi_1+\phi_2}{2}\right)
		\cos\frac{\Delta{}l\cdot\omega}{2C}
\end{equation*}

При построении спектра сигнала составляющая
$\sin\left(\omega{}t+\frac{\phi_1+\phi_2}{2}\right)$ является копией оригинального
сигнала, сдинутого по фазе. При любых значений $\phi_1$ и $\phi_2$ спектр этого
сдвинутого сигнала совпадает с исходным. Другая составляющая ---
$\cos\frac{\Delta\phi}{2}$ --- зависит только от разности фаз.
А разность фаз в свою очередь является функцией частоты исходного сигнала.
Поэтому при расчете спектра эта составляющая также  превращается в функцию
частоты.

Определим спектры сигналов на частоте $\omega=2\pi{}f$:
\begin{equation*}
	\begin{aligned}
		F_{s_1(t)} &= \frac{1}{\sqrt{2\pi}}\int\limits_{-\infty}^\infty
	                  \frac{1}{2i}\left(e^{i\omega{}t+i\phi_1}-e^{-i\omega{}t-i\phi_1}\right)\cdot{}e^{-i\omega{}t}dt=\\
	               &= \frac{1}{\sqrt{2\pi}}\int\limits_{-\infty}^\infty
	                  \frac{1}{2i}\left(e^{i\omega{}t+i\phi_1}\cdot{}e^{-i\omega{}t}-e^{-i\omega{}t-i\phi_1}\cdot{}e^{-i\omega{}t}\right)dt=\\
	               &= \frac{1}{\sqrt{2\pi}}\int\limits_{-\infty}^\infty
	                  \frac{1}{2i}\left(e^{i\omega{}t+i\phi_1-i\omega{}t}-e^{-i\omega{}t-i\phi_1-i\omega{}t}\right)dt=\\
	               &= \frac{1}{\sqrt{2\pi}}\int\limits_{-\infty}^\infty
	                  \frac{1}{2i}\left(e^{i\phi_1}-e^{-i(2\omega{}t+\phi_1)}\right)dt=\\
	               &= \frac{1}{i\sqrt{8\pi}}\int\limits_{-\infty}^\infty
	                  e^{i\phi_1}dt -
	                  \frac{1}{i\sqrt{8\pi}}\int\limits_{-\infty}^\infty
	                  e^{-i(2\omega{}t+\phi_1)}dt=\\
	               &= \frac{e^{i\phi_1}}{i\sqrt{8\pi}}\int\limits_{-\infty}^\infty
	                  dt -
	                  \frac{e^{-i\phi_1}}{i\sqrt{8\pi}}\int\limits_{-\infty}^\infty
	                  e^{-2i\omega{}t}dt
	\end{aligned}
\end{equation*}
\begin{equation*}
	F_{s_2(t)} = \frac{e^{i\phi_2}}{i\sqrt{8\pi}}\int\limits_{-\infty}^\infty
                  dt -
                  \frac{e^{-i\phi_2}}{i\sqrt{8\pi}}\int\limits_{-\infty}^\infty
                  e^{-2i\omega{}t}dt
\end{equation*}
\begin{equation*}
	\begin{aligned}
		F_{s_1(t)+s_2(t)} &=
		             \frac{e^{i\phi_1}}{i\sqrt{8\pi}}\int\limits_{-\infty}^\infty dt
		             - \frac{e^{-i\phi_1}}{i\sqrt{8\pi}}\int\limits_{-\infty}^\infty
	                 e^{-2i\omega{}t}dt + 
	                 \frac{e^{i\phi_2}}{i\sqrt{8\pi}}\int\limits_{-\infty}^\infty dt
		             - \frac{e^{-i\phi_2}}{i\sqrt{8\pi}}\int\limits_{-\infty}^\infty
	                 e^{-2i\omega{}t}dt=\\
	                 &=\left(\frac{e^{i\phi_1}}{i\sqrt{8\pi}}+\frac{e^{i\phi_2}}{i\sqrt{8\pi}}\right)
	                 \int\limits_{-\infty}^\infty dt
	                 - \left(\frac{e^{-i\phi_1}}{i\sqrt{8\pi}} +
	                         \frac{e^{-i\phi_2}}{i\sqrt{8\pi}}\right)
	                 \int\limits_{-\infty}^\infty{}e^{-2i\omega{}t}dt=\\
	                 &=\frac{e^{i\phi_1}+e^{i\phi_2}}{i\sqrt{8\pi}}
	                 \int\limits_{-\infty}^\infty dt
	                 - \frac{e^{-i\phi_1}+e^{-i\phi_2}}{i\sqrt{8\pi}}
	                 \int\limits_{-\infty}^\infty{}e^{-2i\omega{}t}dt=\\
	\end{aligned}
\end{equation*}

Выразим $\phi_2=\phi_1-\Delta\phi$ и упростим числители дробей:
\begin{equation*}
	e^{i\phi_1}+e^{i\phi_2}=e^{i\phi_1}+e^{i\phi_1-i\Delta\phi}=e^{i\phi_1}+e^{i\phi_1}e^{-i\Delta\phi}=e^{i\phi_1}\left(1+e^{-i\Delta\phi}\right)
\end{equation*}
\begin{equation*}
	e^{-i\phi_1}+e^{-i\phi_2}=e^{-i\phi_1}+e^{-i\phi_1+i\Delta\phi}=e^{-i\phi_1}+e^{-i\phi_1}e^{i\Delta\phi}=e^{-i\phi_1}\left(1+e^{i\Delta\phi}\right)
\end{equation*}

Проанализируем составные части выражений (интегралы). Значение интеграла
$\int\limits_{-\infty}^\infty dt$ стремится к бесконечности, тогда как
абсолютное значение интеграла $\int\limits_{-\infty}^\infty{}e^{-2i\omega{}t}dt$
не превышает 2 (сумма синуса и косинуса). Таким образом, при определении
спектральной плотности мощности (делении на диапазон $[-\infty\dots{}\infty]$)
значение первого интеграла сойдется к действительному числу, а второго - к нулю.
Поэтому вторыми интегралами можно пренебречь и переписать формулы в следующем
виде:

\begin{equation*}
	F_{s_1(t)} \approx
                 \frac{e^{i\phi_1}}{i\sqrt{8\pi}}\int\limits_{-\infty}^\infty{}dt
\end{equation*}
\begin{equation*}
	\begin{aligned}
		F_{s_1(t)+s_2(t)} & \approx
	                 \frac{e^{i\phi_1}+e^{i\phi_2}}{i\sqrt{8\pi}}
	                 \int\limits_{-\infty}^\infty dt =
	                 \frac{e^{i\phi_1}\left(1+e^{-i\Delta\phi}\right)}{i\sqrt{8\pi}}
	                 \int\limits_{-\infty}^\infty dt = 
	                 \left(1+e^{-i\Delta\phi}\right)\frac{e^{i\phi_1}}{i\sqrt{8\pi}}
	                 \int\limits_{-\infty}^\infty dt = \\
	               &=\left(1+\cos\Delta\phi-i\sin\Delta\phi\right)\frac{e^{i\phi_1}}{i\sqrt{8\pi}}
	               \int\limits_{-\infty}^\infty dt
	\end{aligned}
\end{equation*}

Видно, что спектры отличаются единственным множителем
\begin{equation*}
K^*(\Delta\phi)=1+\cos\Delta\phi-i\sin\Delta\phi
\end{equation*}
Этот множитель выражает изменения в амлитуде и фазе суммарного сигнала
относительно одного из исходных.

\begin{equation}
	\boxed{
		K=1+\cos\frac{2\pi{}f\cdot\Delta{}l}{C}-i\sin\frac{2\pi{}f\cdot\Delta{}l}{C} =
		1+e^{-i\frac{2\pi{}f\cdot\Delta{}l}{C}} }\label{amplitude_link_k}
\end{equation}

Формула~\ref{amplitude_link_k} связывет амплитуды оригинального и суммарного
сигналов. Эта формула может применяться для сравнения сигналов и определения направления
на источник. Абсолютное значение разности фаз определяет частоту колебаний в
спектра разностного сигнала. Для определения модуля $\Delta{}l$ достаточно
проанализировать действительную часть формулы~\ref{amplitude_link_k}. Построим
спектр $H(\Delta{}l_v)$, где $\Delta{}l_v$ --- изменяемое значение разности фаз

\begin{equation*}
	H(\Delta{}l_v) = \int\limits_{-\infty}^{\infty}
	\left(1+e^{-i\frac{2\pi{}\cdot\Delta{}l}{C}f}\right)e^{-i\frac{2\pi{}\cdot\Delta{}l_v}{C}f}
	df
\end{equation*}

В этой формуле $\frac{2\pi{}\cdot\Delta{}l}{C}$ выступают в роли круговой
частоты $\omega_K(\Delta{}l)$. При работе с спектром, полученным из дискретного
сигнала с использованием БПФ формула должна быть преобразована на основании
следующих рассуждений.

1. Частота сигнала ограничена значениями $\left[-F_d/2\dots{}F_d/2\right]$

2. Операция интегрирования должна быть заменена сумированием.

3. Комплексный множитель формируется из частоты колебаний
$f_K\left(\Delta{}l_v\right)$ амплитуды отношения спектров и частоты
дискретизации сигнала.

\begin{equation*}
	H(\Delta{}l_v) = \sum_{f=-F_d/2}^{F_d/2}
	\left(1+e^{-i\frac{2\pi{}\cdot\Delta{}l}{C}f}\right) 
	e^{-i\cdot{}2\pi{}\frac{f_K\left(\Delta{}l_v\right)}{F_d} f}
\end{equation*}

Спектр будет принимать максимальные значения при совпадении круговых частот
колебания отношения спектров и комплексного множителя. Это выполняется при
выполнении условия
$\frac{f_K\left(\Delta{}l_v\right)}{F_d}=\frac{\Delta{}l}{C}$, откуда

\begin{equation*}
	f_K\left(\Delta{}l_v\right)=f_K\left(\Delta{}l\right)=\frac{\Delta{}l\cdot{}F_d}{C}
\end{equation*}

Определим пределы, в которых достаточно изменять значение $F_K$. Очевидно, что
при правильной работе аппаратуры разность хода фазового фронта не может
превысить расстояние между микрофонами, т.е. $\Delta{}l\le{}S$. Отсюда

\begin{equation*}
	0\le{}f_K\le\frac{S\cdot{}F_d}{C}
\end{equation*}

Для пересчета частоты колебания отношения спектров в разность хода фазового
фронта можно использовать формулу

\begin{equation*}
	\Delta{}l = \frac{f_K\cdot{}C}{F_d}
\end{equation*}

Используя формулу~(\ref{cos_alpha_SL}) можно рассчитать направление на
предполагаемый источник звука.

\end{document}
